\documentclass{report}
\usepackage{amsthm}
\usepackage{enumitem}

\newtheorem*{axiom}{Axiom}

\setlength{\parindent}{4em}
\setlength{\parskip}{1em}
\renewcommand{\baselinestretch}{1.5}

\begin{document}
\title{Controlling Your Life}
\author{Johnson Zhong}

\maketitle

\chapter*{Preface}
\addcontentsline{toc}{chapter}{Preface}

This handbook is my attempt to form a theory on how to live life.
First, I present what we generally want out of life, then the rest of the book
describes how to maximize that. I also consider topics like morality
and compassion in the context of this theory.

\chapter{What We Live For}
\begin{axiom}
	We live to maximize $\int_{birth}^{death} happiness \, dt$
\end{axiom}

Let's call happiness the $h$ function. It has some properties:
\begin{enumerate}[label=H\arabic*]
	\item Happiness is transient and changes with time $\rightarrow$ it's
	a function of time $h(t)$
	\item \label{bounded} We can only be so happy (or sad) at any time $\rightarrow h(t)$ is bounded
	\item \label{positive} Sometimes we feel life is not worth living $\rightarrow$ let this be when $h(t) < 0$
\end{enumerate}  

This axiom is motivated by asking what the meaning of life is.
There's no satisfactory answer because we can always ask why.
Being happy seems to be the most common and the one that
other answers can be broken down to.

\chapter{Sustaining Happiness}
From property \ref{bounded}, it's clear that we need to sustain happiness. 
This rules out going for intense, short bursts through drugs, 
which offer diminishing returns in addition to shortening your lifespan. 

From property \ref{positive}, you'll need to want to remain alive as long as
possible. This means exercising and staying healthy so living is pleasant. 

\section{Happiness Stability}
Regardless of what your sources of happiness are, you should analyze
each in terms of how stable it is in time, and what could take it away.

Many people will find that they rely heavily on social interactions and the
affection of others for their happiness. This source doesn't scale very well 
since relationships have to be maintained at constant upkeep, 
and people could easily change their disposition towards you.
Furthermore, they could even hold your need for approval hostage, leading to
abusive relationships. You really have no control over this source of happiness
- you can't force someone to like you.


On the other hand, useful skills and knowledge such as speaking a 
foreign language or knowing how to program is entirely within your control.
Acquiring them depends on your environment, parts of which you can't control,
but for the most part they depend on your effort. Once acquired, they give
you a greater sense of agency and \textbf{empower you to do more in life}.
These sources scale very well because the more of them you have, the more
you're able to achieve accomplishments, another stable source of happiness.

You should care about and prioritize acquiring stable sources of happiness over
unstable ones.

\end{document}